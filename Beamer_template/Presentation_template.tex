%%%%%%%
\documentclass[aspectratio=169]{beamer}
\usetheme{metropolis}           % Use metropolis theme
%%%%%%%
\usepackage[utf8]{inputenc}
\usepackage[T1]{fontenc}
\usepackage{mathabx}
\usepackage{mathpazo}
\usepackage{eulervm}
\usepackage{adjustbox}
\usepackage{natbib}
\usepackage{multicol}
\usepackage{listings}
\usepackage{helvet}
\usepackage{booktabs}
\usepackage{threeparttable}

\lstset{ %
  backgroundcolor=\color{white},
  basicstyle=\tiny,
  breakatwhitespace=false,
  breaklines=true,
  captionpos=b,
  commentstyle=\color{mygreen},
  escapeinside={\%*}{*)},
  extendedchars=true,
  frame=single,
  numbers=left,
  numbersep=5pt,
  numberstyle=\tiny\color{mygray},
  rulecolor=\color{black},
  showspaces=false,
  showstringspaces=false,
  showtabs=false,
  stringstyle=\color{mymauve},
  tabsize=2,
  title=\lstname,
  morekeywords={not,\},\{,preconditions,effects },
  deletekeywords={time}
}

\title{A workflow for exporting tables from Stata}
\date{\today}
\author{Prepared by DIME Analytics \newline \url{dimeanalytics@worldbank.org}}
\institute{Presented by: \newline Luiza Cardoso de Andrade (\url{lcardoso@worldbank.org}) \newline Benjamin Daniels (\url{bdaniels@worldbank.org})}

\begin{document}
\maketitle

%%%%%%%%%%%%%%%%%%%%%%%%%%%%%%%%%%%%%%%%%%%
\section{Conceptual Framework}

%%%SLIDE%%%
\begin{frame}[fragile]{Main idea: tables in two stages}
Tables have two ``stages'':
\begin{itemize}
    \item In Stage One, you only need \textbf{information}
    \item In Stage Two, you also need \textbf{formatting}
    \item Moving from Step One to Step Two is a big task, that is better to do \textbf{once}
    \item Deciding what stage a table is in will save you a lot of work
    \item But the PI or other teammates need to agree
\end{itemize}
\end{frame}

%%%SLIDE%%%
\begin{frame}[fragile]{Which stage are we in?}
Core questions:
\begin{itemize}
    \item Do I need this output to be immediately shareable without postprocessing?
    \item Do I need this output to be useful for publication, or just for reading results?
    \item Do I need to be able to adjust number formatting and rounding later?
    \item Will I need to adjust table layout and formatting later?
    \item What will be the required workflow when I re-produce this table?
    \item What if I alter models, parameters, or other core components?
\end{itemize}
\end{frame}

%%%SLIDE%%%
\begin{frame}[fragile]{Which stage are we in?}
\begin{table}[]
\begin{tabular}{p{4.5cm}cc}
\hline
                                       & \textbf{Stage One}                                                    & \textbf{Stage Two}        \\
                                       \hline
\textbf{Amount of coding}                       & Little                                                       & Moderate to lots \\
\textbf{Replicability}                          & \begin{tabular}{p{4cm}}Results replicate but may require copying or post-processing \end{tabular} & Fully replicable \\
\textbf{Adjustability of models and parameters} & High                                                         & Moderate to low  \\
\textbf{Suitability to frequent updates}        & Slow                                                         & Fast             \\
\textbf{Formatting}                             & Little to none                                               & Complete        \\
\hline
\end{tabular}
\end{table}
\end{frame}

%%%SLIDE%%%
\begin{frame}[fragile]{Stage One: Focus on the information}
A table in Stage One needs to only follow a few rules:
\begin{itemize}
    \item One file per output, named functionally, like \texttt{regression-robustness.xlsx}
    \item It should require \textbf{zero} formatting by hand for the core results to be readable
    \item It should be created with built-in or user-written Stata commands, unless impossible
    \item It should be created in a standalone code chunk beginning with \texttt{use}
\end{itemize}
What do these requirements accomplish? They have one goal: when you need to see how something changes,
you simply re-run this table's specific code chunk, and new results are instantly available.
\end{frame}

%Add instructions to create a matrix from results

%%%SLIDE%%%
\begin{frame}[fragile]{Moving to Stage Two: Focus on the structure}
Only move a table to Stage Two when you are 100\% (ideally -- 85\% is more realistic) decided on the \textbf{structure}.

Structure is anything that would be \textbf{hard to change} in code, such as:
\begin{itemize}
    %\item Placement in the paper (such as the name of the table)
    \item Exactly which rows and columns need to be displayed and in what order
    \item Exactly how any text & decoration should be implemented
\end{itemize}
These are often things that non-coders think are easy to change -- they may think it is as easy as moving things in Excel. But these are places where you can waste a lot of time being asked to make ``little'' changes if it's not clear that these are costly things to change.
\end{frame}

%%%SLIDE%%%
\begin{frame}[fragile]{Stage Two: Focus on formatting}
Once a table is in Stage Two, you should have done the following:
\begin{itemize}
    \item One file per output, named structurally, like \texttt{table-01\_regression-robustness.xlsx}
    \item Have set up formatting either in code, or such that only \textbf{one} copy-paste operation in Excel is needed to fully recreate the new table
    \item The exact numbers and stars should appear in exactly the right places (this may require custom coding such as building up a matrix)
    \item It should be created in a standalone code chunk beginning with \texttt{use}
\end{itemize}
These requirements are very similar to those in Stage One, except once you have put in the extra work formatting your table, the re-run is now able to create a production-ready output.
\end{frame}

\begin{frame}[fragile]{Writing an arbitrary matrix: Review}
\begin{lstlisting}
// Use data
sysuse census.dta , clear

// Loop over variables
local indepvars death marriage divorce
local depvars pop poplt5 pop5_17 pop65p pop18p popurban

cap mat drop results results_STARS  // Gotta do this
qui foreach depvar in `depvars' {
  reg `depvar' `indepvars'          // Run regression
  mat a = r(table)                  // Store results
  mat a = a[....,1]'                // Get [death] only and transpose
  mat rownames a = "`depvar'"       // Proper label
  mat results = nullmat(results)   /// First run
    \ a                             // Stack up results
}

// Print this matrix nicely: Excel
outwrite results /// If only one name, matrix is assumed
  using "/path/to/file.xlsx" , replace
  
// Print this matrix nicely: LaTeX
\end{lstlisting}
\end{frame}

%%%%%%%%%%%%%%%%%%%%%%%%%%%%%%%%%%%%%%%%%%%
\section{How to make it relatively painless: Excel}

%%%SLIDE%%%
\begin{frame}[fragile]{Run and store some regressions}
\begin{multicols}{2}
\begin{lstlisting}
// Data
sysuse census.dta, clear

// Basic Regression
reg divorce marriage pop
    est sto reg1

// Basic Regression 2
reg medage popurban
    est sto reg2

// Indicator Regression
reg divorce marriage pop i.region
    est sto reg3

// Interaction Regression
gen binary = rnormal() > 0
    lab def binary 0 "No" 1 "Yes"
    lab val binary binary
    label var binary "Indicator"
reg divorce marriage pop i.region#i.binary
    est sto reg4
\end{lstlisting}
\parbox{\linewidth}{
In this code we have four of the core functions of regressions: basic estimates, different models, and indicator and interaction expansion.
\newline \newline
From these specifications you can make almost any of the typical IE estimators.
\newline \newline
This can be a lot of information, and Stata is not very good at displaying multiple regressions simultaneously.
}
\end{multicols}
\end{frame}

%%%SLIDE%%%
\begin{frame}[fragile]{Export to Stage One using four user-written programs}
\begin{multicols}{2}
\begin{lstlisting}
// outreg2
outreg2 [reg1 reg2 reg3 reg4]   ///
using "outputs/outreg.xls"      ///
, replace excel

// estout
estout reg1 reg2 reg3 reg4      ///
using "outputs/estout.xls"      ///
, replace c(b & _star se)

// xml_tab
xml_tab reg1 reg2 reg3 reg4     ///
, save("outputs/xml_tab.xls")   ///
  replace below

// outwrite
outwrite reg1 reg2 reg3 reg4    ///
using "outputs/outwrite.xlsx"   ///
, replace
\end{lstlisting}
\parbox{\linewidth}{
Each of these four commands has different features. The first three have lots of formatting options, but you should use these sparingly.
\newline \newline
Notice how they take different approaches to (a) creating the file and (b) handling the four different regressions and their features together.
\newline \newline
The last one is new (from us), and is intended to take the best bits of these commands and make it easier to output a simple table.
}
\end{multicols}
\end{frame}

%%%SLIDE%%%
\begin{frame}[fragile]{What does Stage One look like? (\texttt{outwrite})}
\begin{multicols}{2}
    Key Excel-related decisions:
    \newline \newline
    Fully XLSX native -- Uses \textbf{Stata 15}'s \texttt{putexcel} to write file
    \newline \newline
    Cells at \textbf{full precision} -- you can use the decimal place adjustment tools within Excel; from Stata only the overall \texttt{format()} can be set
    \newline \newline
    Stars are \textbf{formatted} -- they will be \textbf{lost} by \texttt{Copy → Paste Special → Values}
\begin{figure}
    \centering
    \includegraphics[width=\linewidth]{outwrite.png}
\end{figure}
\end{multicols}
\end{frame}

%%%SLIDE%%%
\begin{frame}[fragile]{What does Stage Two look like?}
\begin{multicols}{2}
     To move to Stage Two:
    \newline \newline
    Set up a \texttt{\_final-tables.xlsx} file with one sheet per table; all the formatting, titles, bolding, column numbering, model notes, etc. can be set up in here
    \newline \newline
    But the alignment of the cells should be such that the raw table will copy \textbf{exactly} into this frame -- re-running all the tables and inserting them here should take at most 15 minutes
\begin{figure}
    \centering
    \includegraphics[width=\linewidth]{final-tables.png}
\end{figure}
\end{multicols}
\end{frame}

\begin{frame}[Stage Two formatting in Excel]
What to do when moving into Stage Two, but is a waste of time in Stage one:
\begin{itemize}
    \item Organize the tables into sheets (by the end, these are \textbf{ordered})
    \item Make the raw outputs "almost" exactly what you need -- so that no "micro" formatting or "micro" copy-pasting needs to be done to update the final table (only "macro" formatting and copy-pasting)
    \item For example, if you have two panels, make \texttt{table-03\_1.xlsx and table-03\_2.xlsx}
    \item The biggest pain is things like italics, parentheses, and bolding -- wait till the very end to do this (or omit entirely)
    \item Use \texttt{Paste Special → Values} and \texttt{Paste Special → Transpose}
    \item Use the Format Paintbrush intelligently (but be careful with decimal places and stars when doing so)
\end{itemize}
\end{frame}

%%%%%%%%%%%%%%%%%%%%%%%%%%%%%%%%%%%%%%%%%%%
\section{How to make it relatively painless: LaTeX}
\begin{frame}[fragile]{Not much new here}
\begin{multicols}{2}
\begin{lstlisting}
// Load the data **************************
sysuse census.dta, clear

// Run regressions ************************
// Regression 1: nothing interesting
reg death marriage pop
est sto reg1
estadd local region "No"

// Regression 2: a different regression
reg death popurban
est sto reg2
estadd local region "No"

// Regression 3: indicator expansion
reg divorce marriage pop
est sto reg3
estadd local region "No"

// Regression 4: interaction
reg divorce marriage pop i.region
est sto reg4
estadd local region "Yes"
\end{lstlisting}

\vspace{2cm}
\vspace{2cm}
\begin{itemize}
    \item Let's start by just running some regressions
    \item \texttt{est sto} saves the regression results
    \item \texttt{estadd} saves additional information to be displayed on the table
\end{itemize}

\end{multicols}
\end{frame}


\begin{frame}[fragile]{Export simple table to \LaTeX }
The simplest possible table is created with the code to the left and looks as displayed on the right.

\begin{multicols}{2}
\begin{lstlisting}
global out_raw "FILE/PATH/HERE"

esttab  reg1 reg2 reg3 reg4 ///
    using "${out_raw}/esttab_label.tex", ///
    label /// Add variable labels
    replace
\end{lstlisting}

\begin{table}
\begin{adjustbox}{max width = .45\textwidth}
\input{tables/esttab_label.tex}
\end{adjustbox}
\end{table}
\end{multicols}
\end{frame}

\begin{frame}[fragile]{Stage one formatting in \LaTeX }
\begin{itemize}
    \item There's still some simple formatting that you can do while in stage 1
    \item This includes formatting that the command can still generalize for other model adjustments that will still be made
    \item Almost all options shown in \texttt{help esttab} fall within this case
\end{itemize}
\end{frame}

\begin{frame}[fragile]{Stage one formatting in \LaTeX }
\begin{multicols}{2}
\begin{lstlisting}
esttab reg1 reg2 reg3 reg4 ///
    using "${out_raw}/esttab_scalar.tex", ///
    scalars("region Region fixed effects")  ///
    addnotes("Add a note here." "Other custom note here.")  ///
    label ///
    ci ///
    replace
\end{lstlisting}

\begin{table}
\begin{adjustbox}{max width = .45\textwidth}
\input{tables/esttab_scalar.tex}
\end{adjustbox}
\end{table}
\end{multicols}
\end{frame}

\begin{frame}[fragile]{Stage one formatting in \LaTeX }
Here's a list of useful options:
\begin{itemize}
\scriptsize
    \item   \texttt{se[(fmt)]}: display standard errors instead of t-statistics
    \item   \texttt{p[(fmt)]}: display p-values instead of t-statistics
    \item   \texttt{ci[(fmt)]}: display confidence intervals instead of t-stat's
    \item   \texttt{[no]constant}: do not/do report the intercept
    \item   \texttt{[no]star[(list)]}: do not/do report significance stars
    \item   \texttt{r2|ar2|pr2[(fmt)]}: display (adjusted, pseudo) R-squared
    \item   \texttt{scalars(list)}: display any other scalars contained in e()
    \item   \texttt{noobs}: do not display the number of observations
    \item   \texttt{label}: make use of variable labels
    \item   \texttt{nomtitles}: disable model titles
    \item   \texttt{[no]depvars}: do not/do use dependent variables as model titles
    \item   \texttt{[no]numbers}: do not/do print model numbers in table header
    \item   \texttt{[no]notes}: suppress/add notes in the table footer
    \item   \texttt{addnotes(list)}: add lines at the end of the table
    \item   \texttt{noomitted}: drop omitted coefficients
    \item   \texttt{nobaselevels}: drop base levels of factor variables
\end{itemize}
\end{frame}

\begin{frame}[fragile]{Moving on to stage two in \LaTeX }

\begin{itemize}
    \item More advanced options than the ones listed in the previous slide can cause you some trouble when you modify your models
    \item This includes options that require you to spell out variable names, embed {\LaTeX} code in Stata code or cite models individually
    \item Let's look at some examples
\end{itemize}
\end{frame}


\begin{frame}[fragile]{Stage two formatting in \LaTeX }
Here are the two easiest way to add formatting that may crash your code if you change your models:

\begin{multicols}{2}
\begin{lstlisting}
esttab reg1 reg2 reg3 reg4 ///
    using "${out_raw}/esttab_titles.tex", ///
    mtitles("Title 1" "Title 2" ///
            "Title 3" "Title 4") ///
    drop(*.region*) ///
    scalars("region Region fixed effects") ///
    addnotes("Add a note here." "Other custom note here.")  ///
    label ///
    replace
\end{lstlisting}

\begin{table}
\begin{adjustbox}{max width = .45\textwidth}
\input{tables/esttab_titles.tex}
\end{adjustbox}
\end{table}
\end{multicols}
\end{frame}

\begin{frame}[fragile]{Stage two formatting in \LaTeX }
Here's what you \textit{really} shouldn't do before you have final models -- the consequence being to risk spending a lot of time on formatting for small edits

\begin{multicols}{2}
\begin{lstlisting}
esttab reg1 reg2 reg3 reg4 ///
    using "${out_raw}/esttab_header.tex", ///
    prehead("\begin{tabular}{l*{4}{c}} \hline\hline & \multicolumn{4}{c}{\textit{Dependent variable:}} \\ & \multicolumn{2}{c}{Number of deaths} & \multicolumn{2}{c}{Number of divorces} \\ \cmidrule(lr){2-3} \cmidrule(lr){4-5} ") ///
    nomtitles ///
    drop(*.region*) ///
    scalars("region Region fixed effects") ///
    label ///
    replace
\end{lstlisting}

\begin{table}
\begin{adjustbox}{max width = .45\textwidth}
\input{tables/esttab_header.tex}
\end{adjustbox}
\end{table}
\end{multicols}
\end{frame}

\begin{frame}[fragile]{Stage two formatting in \LaTeX }
And here's what we think is probably the most complicated thing you can do while still using \texttt{esttab}

\begin{lstlisting}
* A table with two panels --------------------------------------------------------------------------------
* Top panel
esttab reg1 reg2 using "${out_raw}/esttab_panel.tex", ///
    refcat(marriage "\\ \multicolumn{3}{c}{\textbf{Panel A: Number of deaths}} \\[-1ex]  ", nolabel) ///
    prehead("\begin{tabular}{l*{2}{c}} \hline\hline" ) ///
    fragment  ///
    nomtitles noobs ///
    panel ///
    label ///
    replace

* Bottom panel
esttab reg3 reg4 using "${out_raw}/esttab_panel.tex", ///
    refcat(marriage "\\ \multicolumn{3}{c}{\textbf{Panel B: Number of divorces}} \\[-1ex]  ", nolabel) ///
    fragment ///
    append ///
    nomtitles nonumbers nolines ///
    prefoot("\hline") ///
    postfoot("\hline\hline \end{tabular} \begin{tablenotes} \footnotesize \item Add notes manually here. \end{tablenotes}") ///
    drop(*.region*) ///
    label
\end{lstlisting}
\end{frame}

\begin{frame}[fragile]{Stage two formatting in \LaTeX }
\begin{table}
\scriptsize
\begin{adjustbox}{max width = .3\textwidth}
\begin{threeparttable}
\input{tables/esttab_panel.tex}
\end{threeparttable}
\end{adjustbox}
\end{table}
\end{frame}

%%%%%%%%%%%%%
\section{Thank you!}
\end{document}
%%%%%%%%%%%%%
