
\documentclass[aspectratio=169]{beamer}
\usetheme{metropolis}           % Use metropolis theme
\usepackage[utf8]{inputenc}
\usepackage{graphicx}
\usepackage{eso-pic}
\usepackage{graphics}
\usepackage{tikz}
\usepackage[export]{adjustbox}
\usepackage{multicol}
\usepackage{listings}
\usepackage{helvet}
\usepackage{booktabs}
\usepackage{threeparttable}
\usepackage{marvosym}
\usepackage{hyperref}

\title{Data Quality Checks}
\date{June 13, 2019}
\author{Luiza Andrade and Matteo Ruzzante} % Name of author(s) of session here
\institute{Development Impact Evaluation (DIME) \newline The World Bank }
\setbeamercolor{background canvas}{bg=white}	% Sets background color

% The below command places the World Bank logo and DIME logo to the right corner
\titlegraphic{%
	\begin{picture}(0,0)
	\put(330,-180){\makebox(0,0)[rt]{\includegraphics[width=3cm]{img/WB_logo}}}
	\end{picture}%
	\begin{picture}(0,0)
	\put(390,-180){\makebox(0,0)[rt]{\includegraphics[width=1.5cm]{img/i2i}}}
	\end{picture}%
}

%%% Section page with picture of Light bulb
\makeatletter
\defbeamertemplate*{section page}{mytheme}[1][]{
	\centering
	\begin{minipage}{22em}
		\raggedright
		\usebeamercolor[fg]{section title}
		\usebeamerfont{section title}
		\par
		\ifx\insertsubsectionhead\@empty\else%
		\usebeamercolor[fg]{subsection title}%
		\usebeamerfont{subsection title}%
		\fi
		\ifstrempty{#1}{}{%
			\includegraphics[width=100mm, height=60mm]{#1}%
		}
		\\
		\insertsectionhead\\[-1ex]
		\insertsubsectionhead
		\usebeamertemplate*{progress bar in section page}
		
	\end{minipage}
	\par
	\vspace{\baselineskip}
}
\makeatother

%%% Define a command to include picture in section, 
%%% make section, and revert to old template
\newcommand{\sectionpic}[2]{
	\setbeamertemplate{section page}[mytheme][#2]
	\section{#1}
	\setbeamertemplate{section page}[mytheme]
}

%%% The command below allows for the text that contains Stata code
\lstset{ %
	backgroundcolor=\color{white},
	basicstyle=\tiny,
	breakatwhitespace=false,
	breaklines=true,
	captionpos=b,
	commentstyle=\color{green},
	escapeinside={\%*}{*)},
	extendedchars=true,
	frame=single,
	numbers=left,
	numbersep=5pt,
	numberstyle=\tiny\color{gray},
	rulecolor=\color{black},
	showspaces=false,
	showstringspaces=false,
	showtabs=false,
	stringstyle=\color{mauve},
	tabsize=2,
	title=\lstname,
	morekeywords={not,\},\{,preconditions,effects },
	deletekeywords={time}
}

%% The below command creates the ligh bulb logos in the top right corner of the 
\begin{document}
	
{
\usebackgroundtemplate{\includegraphics[height=55mm,right]{img/top_right_corner.pdf}} 
\maketitle
}
%%%%%%%%%%%%%%%%%%%%%%%%%%%%%%%%%%%%%%%%%%% heading of section 1

\begin{frame}{Motivation}

\begin{itemize}
    \setlength\itemsep{1.25em}
    \item Data quality is a fundamental ingredient of a sound impact evaluation \\ and should be assured through \textbf{daily monitoring}
    \item As this is a quite time demanding activity, we can set up some ``flows'' to partially automatize it (and help us not give up during data collection...)
    \pause
    \\[2.5em]
    \item Quality checks can be directly embedded in the SurveyCTO form but most of the times this not enough
    \item Two additional tools are \textit{high frequency checks} (HFCs) and \textit{backchecks} (BCs) %today we are going to focus on how to follow up on these, assuming they were properly set up before the start of data collection
\end{itemize}

\end{frame}

%%%%%%%%%%%%%%%%%%%%%%%%%%%%%%%%%%%%%%%%%%% heading of section 1
\sectionpic{Overview}{img/section_slide}

\begin{frame}{How to Organize the Daily Flow}

\begin{itemize}
    \setlength\itemsep{1.4em}
    
    \item Do-files to run HFCs (in one click) and choose BCs need to be prepared in advance and ideally be tested during the pilot or the training
    
    \item \textbf{HFCs} should flag every case where there are
    \begin{itemize}
        \item Missing data (e.g., GPS)
        \item Clear typos (e.g., wrong magnitude)
        \item Extreme values (e.g., revenues or yields)
        \item Inconsistent answers across the survey (e.g., sold more than produced)
        \item New household name
    \end{itemize}
    \vspace{-1.25em}
    \item and save these information in a daily report
    
    \pause
    \item In addition, a subsample of interviewees (ideally 10\%) should be subjected to a shorter version of the survey (\textbf{BC}) to check the legitimacy of the original data
    
\end{itemize}
\end{frame}

\begin{frame}{How to Communicate with the Field Team}
    
    \begin{itemize}
    \setlength\itemsep{2em}
        
        \vspace{1em}
        \item Need to set up a direct line between the field team and the FC/RA who will be doing HFCs
        \begin{itemize}
            \item Division of tasks and frequent communication is key
        \end{itemize}
        \item HFC flags can be shared with the field team through DropBox folders
        \begin{itemize}
            \item Or Google Sheet, but issue with connections
        \end{itemize}
        \item The list of new BCs should be picked by the FC/RA and directly sent to the field team at the end of each day
        \begin{itemize}
            \item Can start with a random sample of interview and then focus on enumerators with higher number of flags
        \end{itemize}
        
    \end{itemize}
    
\end{frame}

\begin{frame}{How to Follow Up}

\begin{itemize}
    \setlength\itemsep{1.5em}
    \item Verify if the enumerator's answer to the flagged issues was satisfying
        \begin{itemize}
            \item Otherwise, request second follow up
        \end{itemize}
    
    \item Serious cases (e.g, lower number of plots in an agricultural survey) may need re-surveying
        \begin{itemize}
            \item It is critical to verify this ASAP since the enumerator is most likely close by
            \item So it is easy to re-interview and get missing data if needed
        \end{itemize}
        
    \item Based on the BCs data, we can ask to re-do either a section or the entire survey
    
\end{itemize}

\end{frame}

%%%%%%%%%%%%%%%%%%%%%%%%%%%%%%%%%%%%%%%%%%% heading of section 2
\sectionpic{Exercise}{img/section_slide}

%%%%%%%%%%%%%%%%%%%%%%%%%%%%%%%%%%%%%%%%%%% Final thougts section

\sectionpic{Conclusion}{img/section_slide}

\begin{frame}{Conclusion}

    \begin{itemize}
        \item Let's take data quality seriously, and do HFCs and BCs!
        \item There is scope to optimally split the work between the office and the field
    \end{itemize}
    
    \vspace{1.25em}
    You can find additional materials in the DIME Wiki page (\textcolor{blue}{\hyperlink{https://dimewiki.worldbank.org/wiki/Monitoring_Data_Quality}{HFCs}} and \textcolor{blue}{\hyperlink{https://dimewiki.worldbank.org/wiki/Back_Checks}{BCs}}) \\ and in the \textcolor{blue}{\hyperlink{https://github.com/PovertyAction/high-frequency-checks}{IPA Stata template}}. \\
    Also see examples of HFC and BC codes used by DIME Mozambique team \textcolor{blue}{\hyperlink{ADD LINK HERE}{here}}.
    
    \vspace{1.25em}
    For more information contact:
    \newline Luiza Andrade (\url{lcardoso@worldbank.org})
    \newline Matteo Ruzzante (\url{mruzzante@worldbank.org}) 

\end{frame}

%%%%%%%%%%%%%%%%%%%%%%%%%%%%%%%%%%%%%%%%%%% The End
\sectionpic{The End}{img/section_slide}

\end{document} 