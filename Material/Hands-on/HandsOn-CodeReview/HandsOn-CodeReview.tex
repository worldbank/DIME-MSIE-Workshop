
\documentclass[aspectratio=169]{beamer}
\usetheme{metropolis}           % Use metropolis theme
\usepackage[utf8]{inputenc}
\usepackage{graphicx}
\usepackage{eso-pic}
\usepackage{graphics}
\usepackage{tikz}
\usepackage[export]{adjustbox}
\usepackage{multicol}
\usepackage{listings}
\usepackage{helvet}
\usepackage{booktabs}
\usepackage{threeparttable}
\usepackage{marvosym}
\usepackage{hyperref}

\title{ieduplicates}
\date{June 14, 2019}
\author{Luiza Andrade, and Matteo Ruzzante} % Name of author(s) of session here
\institute{Development Impact Evaluation (DIME) \newline The World Bank }
\setbeamercolor{background canvas}{bg=white}	% Sets background color

% The below command places the World Bank logo and DIME logo to the right corner
\titlegraphic{%
	\begin{picture}(0,0)
	\put(330,-180){\makebox(0,0)[rt]{\includegraphics[width=3cm]{img/WB_logo}}}
	\end{picture}%
	\begin{picture}(0,0)
	\put(390,-180){\makebox(0,0)[rt]{\includegraphics[width=1.5cm]{img/i2i}}}
	\end{picture}%
}

%%% Section page with picture of Light bulb
\makeatletter
\defbeamertemplate*{section page}{mytheme}[1][]{
	\centering
	\begin{minipage}{22em}
		\raggedright
		\usebeamercolor[fg]{section title}
		\usebeamerfont{section title}
		\par
		\ifx\insertsubsectionhead\@empty\else%
		\usebeamercolor[fg]{subsection title}%
		\usebeamerfont{subsection title}%
		\fi
		\ifstrempty{#1}{}{%
			\includegraphics[width=100mm, height=60mm]{#1}%
		}
		\\
		\insertsectionhead\\[-1ex]
		\insertsubsectionhead
		\usebeamertemplate*{progress bar in section page}
		
	\end{minipage}
	\par
	\vspace{\baselineskip}
}
\makeatother

%%% Define a command to include picture in section, 
%%% make section, and revert to old template
\newcommand{\sectionpic}[2]{
	\setbeamertemplate{section page}[mytheme][#2]
	\section{#1}
	\setbeamertemplate{section page}[mytheme]
}

%%% The command below allows for the text that contains Stata code
\lstset{ %
	backgroundcolor=\color{white},
	basicstyle=\tiny,
	breakatwhitespace=false,
	breaklines=true,
	captionpos=b,
	commentstyle=\color{green},
	escapeinside={\%*}{*)},
	extendedchars=true,
	frame=single,
	numbers=left,
	numbersep=5pt,
	numberstyle=\tiny\color{gray},
	rulecolor=\color{black},
	showspaces=false,
	showstringspaces=false,
	showtabs=false,
	stringstyle=\color{mauve},
	tabsize=2,
	title=\lstname,
	morekeywords={not,\},\{,preconditions,effects },
	deletekeywords={time}
}

%% The below command creates the ligh bulb logos in the top right corner of the 
\begin{document}
	
	{
		\usebackgroundtemplate{\includegraphics[height=55mm,right]{img/top_right_corner.pdf}} 
		\maketitle
	}
	
\begin{frame}{Why do code review?}
	\textit{Code review} is an exercise where one person checks a script written by someone else by reading and running it. The purpose of code review is to improve coding skills and learn from your peers, including	
	\begin{itemize}
		\item Identifying errors
		\item Improving code quality and readability
		\item Improving efficiency
		\item Identifying security issues
		\item Learning different ways of performing a task
		\item Guaranteeing code replicability
	\end{itemize}
\end{frame}

\begin{frame}{How to do code review?}
\begin{enumerate}
	\item Partner up with someone who's familiar with the language you're coding
	\item Share the review package for a script you've been working on in a thumb-drive, including
	\begin{itemize}
		\item The master script
		\item The script to be review (no more than 400 lines of code)
		\item The (de-identified) data necessary to run the script
		\item Any outputs to be recreated 
	\end{itemize} 
	\item Explain to your partner the purpose of the script and give any relevant information for them to understand the task being performed
	\item Read and run the code shared by your partner. Use \texttt{Code review form.xlsx} for guidelines on what to look for and documentation of suggestions
	\item If part of the code is not running, or you have a question about the code, ask your partner
\end{enumerate}
\end{frame}



\end{document} 