
\documentclass[aspectratio=169]{beamer}

\usetheme{metropolis}           % Use metropolis theme
\usepackage[utf8]{inputenc}
\usepackage{graphicx}
\usepackage{eso-pic}
\usepackage{graphics}
\usepackage{tikz}
\usepackage[export]{adjustbox}
\usepackage{multicol}
\usepackage{listings}
\usepackage{helvet}
\usepackage{booktabs}
\usepackage{threeparttable}
\usepackage{marvosym}
\usepackage{hyperref}

\title{ieduplicates}
\date{June 14, 2019}
\author{Luiza Andrade, and Matteo Ruzzante} % Name of author(s) of session here
\institute{Development Impact Evaluation (DIME) \newline The World Bank }
\setbeamercolor{background canvas}{bg=white}	% Sets background color

% The below command places the World Bank logo and DIME logo to the right corner
\titlegraphic{%
	\begin{picture}(0,0)
	\put(330,-180){\makebox(0,0)[rt]{\includegraphics[width=3cm]{img/WB_logo}}}
	\end{picture}%
	\begin{picture}(0,0)
	\put(390,-180){\makebox(0,0)[rt]{\includegraphics[width=1.5cm]{img/i2i}}}
	\end{picture}%
}

%%% Section page with picture of Light bulb
\makeatletter
\defbeamertemplate*{section page}{mytheme}[1][]{
	\centering
	\begin{minipage}{22em}
		\raggedright
		\usebeamercolor[fg]{section title}
		\usebeamerfont{section title}
		\par
		\ifx\insertsubsectionhead\@empty\else%
		\usebeamercolor[fg]{subsection title}%
		\usebeamerfont{subsection title}%
		\fi
		\ifstrempty{#1}{}{%
			\includegraphics[width=100mm, height=60mm]{#1}%
		}
		\\
		\insertsectionhead\\[-1ex]
		\insertsubsectionhead
		\usebeamertemplate*{progress bar in section page}
		
	\end{minipage}
	\par
	\vspace{\baselineskip}
}
\makeatother

%%% Define a command to include picture in section, 
%%% make section, and revert to old template
\newcommand{\sectionpic}[2]{
	\setbeamertemplate{section page}[mytheme][#2]
	\section{#1}
	\setbeamertemplate{section page}[mytheme]
}

%%% The command below allows for the text that contains Stata code
\lstset{ %
	backgroundcolor=\color{white},
	basicstyle=\tiny,
	breakatwhitespace=false,
	breaklines=true,
	captionpos=b,
	commentstyle=\color{green},
	escapeinside={\%*}{*)},
	extendedchars=true,
	frame=single,
	numbers=left,
	numbersep=5pt,
	numberstyle=\tiny\color{gray},
	rulecolor=\color{black},
	showspaces=false,
	showstringspaces=false,
	showtabs=false,
	stringstyle=\color{mauve},
	tabsize=2,
	title=\lstname,
	morekeywords={not,\},\{,preconditions,effects },
	deletekeywords={time}
}

%% The below command creates the ligh bulb logos in the top right corner of the 
\begin{document}
	
{
\usebackgroundtemplate{\includegraphics[height=55mm,right]{img/top_right_corner.pdf}} 
\maketitle
}

%%%%%%%%%%%%%%%%%%%%%%%%%%%%%%%%%%%%%%%%%%% heading of section 3
\sectionpic{ieduplicates}{img/section_slide}

\begin{frame}{ieduplicates}
    \begin{itemize}
        \item \texttt{ieduplicates} is used to identify and resolve duplicates in raw survey data
        \item It's part of DIME Analytics' \texttt{iefieldkit} package
        \item The command outputs a report of all the duplicated entries of a variable (in Excel), and removes the duplicates from the data set until they are resolved
        \item The Excel report is used to document the cases of duplicated interviews and how they were solved
    \end{itemize}
\end{frame}

\begin{frame}{Objectives}
\begin{itemize}
	\item Learn how to use \texttt{ieduplicates}' basic options
	\item Understand how to solve common causes of duplicated entries
	\item Use \texttt{ieduplicates}' duplicates report
\end{itemize}
\end{frame}

\begin{frame}{Keep in mind}
    It is important to solve duplicates during the data collection
    \begin{itemize}
        \item To guarantee that the desired sample size is achieved
        \item To be able to run other quality checks that require unique IDs
        \item More importantly, you may not remember what happened later, and may not be able to come back to verify the information
    \end{itemize}
\end{frame}

\begin{frame}{Why do we have duplicates?}
    \begin{enumerate}
        \item Double submissions of same observation and the same data
         \begin{itemize}
            \item First upload from tablet not complete due to bad internet
        \end{itemize}
        \item Double submissions of same observation but with modified data
         \begin{itemize}
            \item Answers modified after submission, and then re-submitted
            \item Bad practice, more transparent to write correction in a do-file
            \item Many survey software have features preventing this
        \end{itemize}
        \item Incorrectly assigned ID. Two respondents are given the same ID
         \begin{itemize}
            \item Typo in the field when entering respondent ID
        \end{itemize}
        \item Household was interviewed twice
        \begin{itemize}
            \item First visit was rescheduled
            \item There was a problem with the data and individual was re-surveyed
        \end{itemize}
    \end{enumerate}   
\end{frame}

\begin{frame}{Exercise 1: Looking for duplicates}
    \begin{enumerate}
        \item Open a new do-file
        \item Start by importing the data set in this session's folder. It's called \texttt{ieduplicates\_dataset.dta}
        \item Use the \texttt{count} command to see how many observations are in your data set
        \item Use the \texttt{isid} command to check if the variable \texttt{uuid} uniquely identifies the data
    \end{enumerate}
\end{frame}

\begin{frame}{Exercise 2: \texttt{ieduplicates} basic syntax}
    \begin{center}
        \large
        \texttt{ieduplicates \textit{ID\_varname}, folder(\textit{string}) uniquevars(\textit{varlist})}    
    \end{center}    
There are other very useful options to the command, but we'll focus on the basic usage. You can see all the different options by typing \texttt{ help ieduplicates}

    \begin{enumerate}
        \item Write the command above to your do-file
        \item Replace \textit{ID\_varname} with \texttt{uuid}: this should be the data set's unique ID
        \item Replace \textit{string} with the path to this session's folder in your computer
        \item Because this dataset was created using SurveyCTO, the unique observation identifier is called \texttt{key}
        \item Run the code. Don't click on the link on your screen yet!
        \item Check how many observations are in your data set now
    \end{enumerate}
\end{frame}

\begin{frame}{Exercise 3: Reading iedupreport}
    \begin{enumerate}
        \item Click on the link to the ieduplicates report. This shows all the duplicated observations in the data set.
        \item It will contain the following columns already filled:
        \begin{itemize}
            \item \texttt{uuid}: the duplicated value of the ID variable
            \item \texttt{duplistid}: used to maintain the sort order in the Excel Report
            \item \texttt{datelisted}: date the duplicated entry was first identified
            \item \texttt{datefixed}: date a valid correction was imported the first time for that duplicated entry
            \item \texttt{key}: the unique observation ID
            \item \texttt{listofdiffs:} a list of all the variables that are different across duplicated entries
        \end{itemize}
    \end{enumerate}
\end{frame}

\begin{frame}{Exercise 4: Solving duplicated entries}
    \begin{enumerate}
        \item Using only the information in \texttt{iedupreport}, which of these issues can you solve?
        \item Open the file in the session's folder called \texttt{Logbook.xlsx}. This is a list made by the field team of all the visits made.
        \item Compare the information in the logbook to the information in the duplicates report and discuss how to solve the duplicated entries still unsolved
    \end{enumerate}
\end{frame}

\begin{frame}{Exercise 4: Solving duplicated entries}
    \begin{enumerate}
        \item To correct the issues, use the blank columns:
        \begin{itemize}
            \item \texttt{correct}: fill in ``yes'' to keep the information as is
            \item \texttt{drop}: fill in ``yes'' to drop the observation
            \item \texttt{newid}: fill in a new ID value to assign to this observatoin
            \item \texttt{initials}: add your initials to keep track on who decided on corrections
            \item \texttt{notes}: write down how the decision was made
        \end{itemize}
        \item Save the duplicates report and close it
        \item Run the do-file that opens the data set and runs \texttt{ieudplicates} again
        \item Check the number of observations in your data set now
    \end{enumerate}
\end{frame}

%%%%%%%%%%%%%%%%%%%%%%%%%%%%%%%%%%%%%%%%%%% Final thougts section

\sectionpic{Thank you!}{img/section_slide}

\begin{frame}{Thank you!}
    
    You can find additional materials in DIME Wiki (\textcolor{blue}{\hyperlink{https://dimewiki.worldbank.org/wiki/Ieduplicates}{ieduplicates}}).
    
    \vspace{0.75em}
    
    If you find any bug, or if you want to submit requests for additional features, please open an issue in the \texttt{iefieldkit} \textcolor{blue}{\hyperlink{https://github.com/worldbank/iefieldkit}{GitHub page}}.
    
    \vspace{1.25em}
    For more information contact:
    \newline Luiza Andrade (\url{lcardoso@worldbank.org})
    \newline Matteo Ruzzante (\url{mruzzante@worldbank.org}) 

\end{frame}

\end{document} 