%%%%%%%%%%%%%%%%%%%%%%%%%%%%%%%%%%%%%%%%%%%%%%%%%%%%%%%%%%%%%%%%%%%%%%
% How to use writeLaTeX:
%
% You edit the source code here on the left, and the preview on the
% right shows you the result within a few seconds.
%
% Bookmark this page and share the URL with your co-authors. They can
% edit at the same time!
%
% You can upload figures, bibliographies, custom classes and
% styles using the files menu.
%
% If you're new to LaTeX, the wikibook is a great place to start:
% http://en.wikibooks.org/wiki/LaTeX
%
%%%%%%%%%%%%%%%%%%%%%%%%%%%%%%%%%%%%%%%%%%%%%%%%%%%%%%%%%%%%%%%%%%%%%%
\documentclass{tufte-handout}

%\geometry{showframe}% for debugging purposes -- displays the margins

\usepackage{amsmath}

% Set up the images/graphics package
\usepackage{graphicx}
\setkeys{Gin}{width=\linewidth,totalheight=\textheight,keepaspectratio}
\graphicspath{{graphics/}}

\title{Hands-On Session: Data Encryption}
\author{DIME Analytics \\ dimeanalytics@worldbank.org}
\date{11 June 2019}  % if the \date{} command is left out, the current date will be used

% The following package makes prettier tables.  We're all about the bling!
\usepackage{booktabs}

% The units package provides nice, non-stacked fractions and better spacing
% for units.
\usepackage{units}

% The fancyvrb package lets us customize the formatting of verbatim
% environments.  We use a slightly smaller font.
\usepackage{upquote}
\usepackage{fancyvrb}
\fvset{fontsize=\normalsize}
\renewcommand{\FancyVerbFormatLine}{\color{violet}}
\DefineShortVerb{\|}

% Small sections of multiple columns
\usepackage{multicol}

% Provides paragraphs of dummy text
\usepackage{lipsum}

% Allows clickable links
\usepackage[colorlinks = true,
linkcolor = blue,
urlcolor  = blue,
citecolor = blue,
anchorcolor = blue]{hyperref}

% These commands are used to pretty-print LaTeX commands
\newcommand{\doccmd}[1]{\texttt{\textbackslash#1}}% command name -- adds backslash automatically
\newcommand{\docopt}[1]{\ensuremath{\langle}\textrm{\textit{#1}}\ensuremath{\rangle}}% optional command argument
\newcommand{\docarg}[1]{\textrm{\textit{#1}}}% (required) command argument
\newenvironment{docspec}{\begin{quote}\noindent}{\end{quote}}% command specification environment
\newcommand{\docenv}[1]{\textsf{#1}}% environment name
\newcommand{\docpkg}[1]{\texttt{#1}}% package name
\newcommand{\doccls}[1]{\texttt{#1}}% document class name
\newcommand{\docclsopt}[1]{\texttt{#1}}% document class option name

\begin{document}

\maketitle% this prints the handout title, author, and date

\begin{marginfigure}%
  \includegraphics[width=\linewidth]{img/light.png}
\end{marginfigure}

\begin{abstract}
In this exercise we will encrypt a dataset and attempt to access the data in an encrypted folder

\bigskip\noindent \textbf{Exercise Objectives}: On completing this exercise you will be able to
\begin{enumerate}
  \item Set up VeraCrypt
  \item Encrypt files
  \item Access encrypted files
\end{enumerate}
\end{abstract}

%\printclassoptions
\section{Part 1: Set up VeraCrypt}

First we will install VeraCrypt. Note that this needs to be done only once. Anyone accessing a file encrypted using VeraCrypt requires the software installed on their computer.

\begin{enumerate}
	\item Install \href{https://www.veracrypt.fr/en/Downloads.html}{Veracrypt}. (eServices request on a Bank desktop/laptop)
	\item Create an encrypted folder (volume) titled 'Encrypted Data' with a strong password.\sidenote{Add suggestions for how to make strong passwords} \\
	\textit{Note that, you can use any drive of your choice.}\sidenote{Add sub-steps for this (maybe with pictures)}
	\item Go to the location where you created the folder to confirm the folder exists.
	Double click on the folder to try and open it directly. Confirm that it doesn't open!
\end{enumerate}

%\printclassoptions
\section{Part 2: Encrypting files}
\begin{enumerate}
	\item Open VeraCrypt.
	\item Mount the encrypted volume created onto drive K. \\
	\textit{Note that, you can use any drive of your choice.}
	\item This will now open an empty folder titled 'Encrypted Data'. Save a copy of the data sets titled endline\_data\_raw\_2018.dta and endline\_data\_raw\_nodup\_2018.dta\sidenote{Update to final dataset once ready} in the folder. \\
	\textit{This is very similar to using a new USB/flash drive and adding files to it.}
	\item Open VeraCrypt and dismount the drive.
	\item Share the password securely with team members.\sidenote{Add something about securely sharing passwords}
	
\end{enumerate}


%\printclassoptions
\section{Part 3: Access encrypted files}
\begin{enumerate}
	\item Open VeraCrypt.
	\item Mount the encrypted volume created onto drive K. \\
	\textit{Note that, you can use any drive of your choice.}
	\item This will now open a folder titled 'Encrypted Data' which contains the data sets titled endline\_data\_raw\_2018.dta and endline\_data\_raw\_nodup\_2018.dta\sidenote{Update to final dataset once ready}.
	\item Open VeraCrypt and dismount the drive   
\end{enumerate}

NOTE: When calling a file in the encrypted folder on Stata the file path used will be that of the mounted drive and not that of where the encrypted file is stored. 
Example: Encrypted folder is stored in Dropbox at\sidenote{add pictures to explain}. It will be used in Stata using mounted drive and not the filepath of the Dropbox folder. add stata example on file path for mounted drive, but do not worry about doing it in the big master dofile
\section{Extra - If time permits}
Pair up with another person in the room and try to create, share, and access encrypted data.
\begin{enumerate}
	\item Create a shared Dropbox/Box/Onedrive folder.
	\item Create one encrypted folder for each of you within this shared folder.
	\item Place the data in your folder in that shared folder.
	\item Share the password with partner
	\item Access partner's encrypted file
\end{enumerate}

\end{document}
