
\documentclass[aspectratio=169]{beamer}
\usetheme{metropolis}           % Use metropolis theme
\usepackage[utf8]{inputenc}
\usepackage{graphicx}
\usepackage{eso-pic}
\usepackage{graphics}
\usepackage{tikz}
\usepackage[export]{adjustbox}
\usepackage{multicol}
\usepackage{listings}
\usepackage{helvet}
\usepackage{booktabs}
\usepackage{threeparttable}
\usepackage{marvosym}

\title{Georeferencing}
\date{June 14, 2019}
\author{Jonas Guthoff and Matteo Ruzzante} % Name of author(s) of session here
\institute{Development Impact Evaluation (DIME) \newline The World Bank }
\setbeamercolor{background canvas}{bg=white}	% Sets background color

% The below command places the World Bank logo and DIME logo to the right corner
\titlegraphic{%
	\begin{picture}(0,0)
	\put(330,-180){\makebox(0,0)[rt]{\includegraphics[width=3cm]{img/WB_logo}}}
	\end{picture}%
	\begin{picture}(0,0)
	\put(390,-180){\makebox(0,0)[rt]{\includegraphics[width=1.5cm]{img/i2i}}}
	\end{picture}%
}

%%% Section page with picture of Light bulb
\makeatletter
\defbeamertemplate*{section page}{mytheme}[1][]{
	\centering
	\begin{minipage}{22em}
		\raggedright
		\usebeamercolor[fg]{section title}
		\usebeamerfont{section title}
		\par
		\ifx\insertsubsectionhead\@empty\else%
		\usebeamercolor[fg]{subsection title}%
		\usebeamerfont{subsection title}%
		\fi
		\ifstrempty{#1}{}{%
			\includegraphics[width=100mm, height=60mm]{#1}%
		}
		\\
		\insertsectionhead\\[-1ex]
		\insertsubsectionhead
		\usebeamertemplate*{progress bar in section page}
		
	\end{minipage}
	\par
	\vspace{\baselineskip}
}
\makeatother

%%% Define a command to include picture in section, 
%%% make section, and revert to old template
\newcommand{\sectionpic}[2]{
	\setbeamertemplate{section page}[mytheme][#2]
	\section{#1}
	\setbeamertemplate{section page}[mytheme]
}

%%% The command below allows for the text that contains Stata code
\lstset{ %
	backgroundcolor=\color{white},
	basicstyle=\tiny,
	breakatwhitespace=false,
	breaklines=true,
	captionpos=b,
	commentstyle=\color{green},
	escapeinside={\%*}{*)},
	extendedchars=true,
	frame=single,
	numbers=left,
	numbersep=5pt,
	numberstyle=\tiny\color{gray},
	rulecolor=\color{black},
	showspaces=false,
	showstringspaces=false,
	showtabs=false,
	stringstyle=\color{mauve},
	tabsize=2,
	title=\lstname,
	morekeywords={not,\},\{,preconditions,effects },
	deletekeywords={time}
}

%% The below command creates the ligh bulb logos in the top right corner of the 
\begin{document}
	
{
\usebackgroundtemplate{\includegraphics[height=55mm,right]{img/top_right_corner.pdf}} 
\maketitle
}
%%%%%%%%%%%%%%%%%%%%%%%%%%%%%%%%%%%%%%%%%%% heading of section 1

\begin{frame}{Motivation}

\begin{itemize}
    \setlength\itemsep{1.5em}
    \item Application in \alert{agricultural studies}: clear location of plots, compute precise plot size, use NDVI values for productivity (vegetation) analysis
    \begin{itemize}
        \vspace{0.75em}
        \setlength\itemsep{0.75em}
        \item Increasingly used for research purposes (plot-level regression analysis, social network, etc.)
        \item Already implemented in some DIME projects (irrigation in Mozambique, technology adoption and diffusion in Burkina Faso)
    \end{itemize}

    \item You can produce some cool pictures for your report or paper \Smiley{}
\end{itemize}

\end{frame}

\begin{frame}{Examples from the Field}
    
    \begin{minipage}{0.5\textwidth}
        \begin{figure}[h!]
            \centering
            \includegraphics[width=6.5cm,height=6.5cm]{fig/Chinhacanine.PNG}
            \caption{Mozambique}
            \label{fig:MOZ}
        \end{figure}
    \end{minipage}%
    \begin{minipage}{0.5\textwidth}
        \begin{figure}[h!]
            \centering
            \includegraphics[width=7cm,height=6.5cm]{fig/Burkina.PNG}
            \caption{Burkina Faso aka Veneto}
            \label{fig:Burkina}
        \end{figure}
    \end{minipage}
    
\end{frame}

%%%%%%%%%%%%%%%%%%%%%%%%%%%%%%%%%%%%%%%%%%% heading of section 1
\sectionpic{SurveyCTO}{img/section_slide}

\begin{frame}{Set Up a \textit{SurveyCTO} Form}

\begin{itemize}
    \item Standard CTO forms take the GPS position of the interview (one point) $\Leftrightarrow$ here we want to take multiple points
    \begin{itemize}
        \item We use the \alert{GPS function} to record latitude, longitude, and altitude with a set accuracy (directly programmed into the form)
        \item We record multiple GPS points in a \alert{repeat} group to generate a polygon
    \end{itemize}
    \item Need to use a bluetooth device to guarantee geo-accuracy %display how it looks like if the GPS is not < 5 metres in the conclusion
   
    \vspace{0.5cm}
    
    \setbeamertemplate{itemize item}[triangle]

    \item \textbf{Let's code the form} %setup a server before, prepare Google Sheet, ask Luiza for attendance list for the preload, which is going to be enumerator list
    \item \textbf{And go get some GPS points out there}:  Form groups and make sure to have at least 1 Android device (tablet or personal phone)
    
\end{itemize}

\end{frame}

%%%%%%%%%%%%%%%%%%%%%%%%%%%%%%%%%%%%%%%%%%% Details for the server

\begin{frame}{Collect Geo Point Data using \textit{SurveyCTO}}
   
    \begin{enumerate}
        \item Download \textit{SurveyCTO collect} from  google play store
    
        \item Setup \textit{SurveyCTO} server on the application and download the Georeferencing (Points)
    
        \begin{center}
            
            \begin{figure}[!htb]
                \minipage{0.2\textwidth}
                \includegraphics[width=\linewidth]{screenshot/main_menu.png}
                \endminipage \hspace{1cm}
                \minipage{0.2\textwidth}%
                \includegraphics[width=\linewidth]{screenshot/menu_settings.png}
                \endminipage \hspace{1cm}
                \minipage{0.2\textwidth}%
                \includegraphics[width=\linewidth]{screenshot/general_settings.png}
                \endminipage
                \label{fig:screenshots}
            \end{figure}
            \FloatBarrier      
        \end{center}        
            
    \end{enumerate}


\end{frame}




\begin{frame}{Collect Geo Point Data using \textit{SurveyCTO} -- Server details}
   

        \begin{itemize}
            \item[3.] The server login details: \par
            \vspace{0.2cm}
            \bf{Server name:} Georeferencing \par
            \bf{Username:} participant \par
            \bf{Password:} fctraining2019 \par
        \end{itemize} 
    
\end{frame}



%%%%%%%%%%%%%%%%%%%%%%%%%%%%%%%%%%%%%%%%%%% Final thougts section

\sectionpic{Mapping}{img/section_slide}

\begin{frame}[fragile]{Mapping}

\begin{itemize}
    
    \item First, we need to reshape the observations to plot and then point level in Stata
    
\begin{lstlisting}
reshape long `gpsVars', i(key point) j(plot)

[...]

reshape long latitude longitude accuracy, i(plot key) j(point) string
\end{lstlisting}


    \item Then, we will use R to generate the spatial polygons and map our plots
    
    \begin{itemize}
        \item Plot polygon(s) with \texttt{ggmap}
        \item Interactive map with \texttt{leaflet}
    \end{itemize}
    
\end{itemize}

\end{frame}

%%%%%%%%%%%%%%%%%%%%%%%%%%%%%%%%%%%%%%%%%%% Final thougts section


\sectionpic{Application \& Conclusion}{img/section_slide}

\begin{frame}{Conclusion}

    \begin{itemize}
        \item Applications for high-frequency checks: were plots well-referenced, compare distances to subjective measures
        \item Potential issues in the field: geo-accuracy is key! %add image comparing plots with and without bluetooth here
    \end{itemize}
    
    \vspace{12.5mm}
    You can find all the materials used today in GitHub.
    
    \vspace{2.5mm}
    For more information contact:
    \newline Matteo Ruzzante (\url{mruzzante@worldbank.org}) \newline Jonas Guthoff (\url{jguthoff@worldbank.org})

\end{frame}

%%%%%%%%%%%%%%%%%%%%%%%%%%%%%%%%%%%%%%%%%%% The End
\sectionpic{The End}{img/section_slide}

\end{document} 