
\documentclass[aspectratio=169]{beamer}
\usetheme{metropolis}           % Use metropolis theme
\usepackage[utf8]{inputenc}
\usepackage{graphicx}
\usepackage{eso-pic}
\usepackage{graphics}
\usepackage{tikz}
\usepackage[export]{adjustbox}
\usepackage{multicol}
\usepackage{listings}
\usepackage{helvet}
\usepackage{booktabs}
\usepackage{threeparttable}


\title{Data Visualization}
\date{\today}
\author{Author of Session here!} % Name of author(s) of session here
\institute{Development Impact Evaluation (DIME) \newline The World Bank }
\setbeamercolor{background canvas}{bg=white}	% Sets background color

% The below command places the World Bank logo and DIME logo to the right corner
\titlegraphic{%
	\begin{picture}(0,0)
	\put(330,-180){\makebox(0,0)[rt]{\includegraphics[width=3cm]{img/WB_logo}}}
	\end{picture}%
	\begin{picture}(0,0)
	\put(390,-180){\makebox(0,0)[rt]{\includegraphics[width=1.5cm]{img/i2i}}}
	\end{picture}%
}

%%% Section page with picture of Light bulb
\makeatletter
\defbeamertemplate*{section page}{mytheme}[1][]{
	\centering
	\begin{minipage}{22em}
		\raggedright
		\usebeamercolor[fg]{section title}
		\usebeamerfont{section title}
		\par
		\ifx\insertsubsectionhead\@empty\else%
		\usebeamercolor[fg]{subsection title}%
		\usebeamerfont{subsection title}%
		\fi
		\ifstrempty{#1}{}{%
			\includegraphics[width=100mm, height=60mm]{#1}%
		}
		\insertsectionhead\\[-1ex]
		\insertsubsectionhead
		\usebeamertemplate*{progress bar in section page}
		
	\end{minipage}
	\par
	\vspace{\baselineskip}
}
\makeatother

%%% Define a command to include picture in section, 
%%% make section, and revert to old template
\newcommand{\sectionpic}[2]{
	\setbeamertemplate{section page}[mytheme][#2]
	\section{#1}
	\setbeamertemplate{section page}[mytheme]
}

%%% The command below allows for the text that contains Stata code
\lstset{ %
	backgroundcolor=\color{white},
	basicstyle=\tiny,
	breakatwhitespace=false,
	breaklines=true,
	captionpos=b,
	commentstyle=\color{green},
	escapeinside={\%*}{*)},
	extendedchars=true,
	frame=single,
	numbers=left,
	numbersep=5pt,
	numberstyle=\tiny\color{gray},
	rulecolor=\color{black},
	showspaces=false,
	showstringspaces=false,
	showtabs=false,
	stringstyle=\color{mauve},
	tabsize=2,
	title=\lstname,
	morekeywords={not,\},\{,preconditions,effects },
	deletekeywords={time}
}

%% The below command creates the ligh bulb logos in the top right corner of the 
\begin{document}
	
{
	\usebackgroundtemplate{\includegraphics[height=55mm, right]{img/top_right_corner.pdf}}
	\maketitle
}


\begin{frame}{Tables give all the details}
	\begin{multicols}{2}	
	
	\begin{itemize}[<default overlay specification>]
		\item<1> What’s happening in this regression table? What’s important?
	\end{itemize}
	
	\begin{figure}
		\centering
		\includegraphics[width=60mm]{img/Table}
	\end{figure}
	
\end{multicols}
\end{frame}

\begin{frame}{But figures tell the story}
	\begin{multicols}{2}	
		
		\begin{itemize}[<default overlay specification>]
			\item<1> This is the data that generates those estimates.
			\item<1> You can see exactly what is happening very quickly!
		\end{itemize}
	
	Even more importantly: You don't have to look for it. The eye is drawn to the story!
		
		\begin{figure}
			\centering
			\includegraphics[width=60mm]{img/Table2}
		\end{figure}
		
	\end{multicols}
\end{frame}


\begin{frame}{Examples: Examining distributions}
	
	\begin{figure}
		\centering
		\includegraphics[width=\linewidth]{img/Distribution}
	\end{figure}
	
\end{frame}

\begin{frame}{Examples: Comparing means}
	
	\begin{figure}
		\centering
		\includegraphics[width=\linewidth]{img/Distribution2}
	\end{figure}
	
\end{frame}


\begin{frame}{Some examples: Comparing correlations}
	
	\begin{figure}
		\centering
		\includegraphics[width=\linewidth]{img/Correlation}
	\end{figure}
	
\end{frame}


\begin{frame}{Some examples: Searching for patterns}
	
	\begin{figure}
		\centering
		\includegraphics[width=\linewidth]{img/Correlation2}
	\end{figure}
	
\end{frame}


\begin{frame}{Some examples: Telling a story about treatment takeup}
	
	\begin{figure}
		\centering
		\includegraphics[width=\linewidth]{img/Correlation3}
	\end{figure}
	
\end{frame}


\begin{frame}{Why do these charts look so good?}
		
		\begin{itemize}[<default overlay specification>]
			\item<1> They tell a story.
			\item<1> They use consistent visual (design) language.
			\item<1> They draw the eye to what is important.
		\end{itemize}
		
\end{frame}


\begin{frame}{Stata default graphs are not very attractive}
	
	\begin{figure}
		\centering
		\includegraphics[width=\linewidth]{img/Graph}
	\end{figure}
	
\end{frame}


\begin{frame}{Stata has three core built-in graph functions}
	\begin{multicols}{2}	
			
			\begin{itemize}[<default overlay specification>]
			\item<1> \textbf{[graph graphtype]}.
				\newline - Graph which plot one or more variables on one axis.
			\item<1> \textbf{[twoway graphtype]}.
				\newline - Graph which two plot variables together on an x,y axis.
			\item<1> \textbf{[histogram kdensity lowess]}.
				\newline - Essential distributional commands.
		\end{itemize}
		
		\begin{figure}
			\centering
			\includegraphics[width=70mm]{img/Function}
		\end{figure}
		
	\end{multicols}
\end{frame}


\begin{frame}{Oneway [graph] plots can be very informative}
	
	\begin{figure}
		\centering
		\includegraphics[width=\linewidth]{img/Graph2}
	\end{figure}
	
\end{frame}


\begin{frame}{[twoway] graphs can be stacked up}
	\begin{multicols}{2}	
		
		\begin{itemize}[<default overlay specification>]
			\item<1> The axes are abstract, so you do not need to use the same variables or the same units for each graph!
			\item<1> Each can have its own if/in and options.
		\end{itemize}
		
		\begin{figure}
			\centering
			\includegraphics[width=70mm]{img/Graph3}
		\end{figure}
		
	\end{multicols}
\end{frame}


\begin{frame}{Charts show information across dimensions}
	\begin{multicols}{2}	
		
		Not these dimensions!
		
		\begin{figure}
			\centering
			\includegraphics[width=70mm]{img/Dimensions}
		\end{figure}
		
		\begin{figure}
			\centering
			\includegraphics[width=70mm]{img/Dimensions2}
		\end{figure}
		
	\end{multicols}
\end{frame}


\begin{frame}{Design with dimensions in mind}
		
		\begin{itemize}[<default overlay specification>]
			\item<1> A chart with a lot of information can blend together like TV static.
		\end{itemize}
		
		\begin{figure}
			\centering
			\includegraphics[width=70mm]{img/Dimensions4}
		\end{figure}
		
\end{frame}


\begin{frame}{Use design language to give charts meaning}
	
	\begin{figure}
		\centering
		\includegraphics[width=\linewidth]{img/Charts}
	\end{figure}
	
\end{frame}


\begin{frame}{Anatomy of most graphs}
	
	\begin{figure}
		\centering
		\includegraphics[width=\linewidth]{img/Charts2}
	\end{figure}
	
\end{frame}


\begin{frame}{Components of Stata graphs}
	
	\begin{figure}
		\centering
		\includegraphics[width=\linewidth]{img/Charts3}
	\end{figure}
	
\end{frame}


\begin{frame}{Best place to start: [h tw]}
	
	\begin{figure}
		\centering
		\includegraphics[width=\linewidth]{img/Graphing}
	\end{figure}
	
\end{frame}


\begin{frame}{Best place to start: [h twoway options]}
	\begin{multicols}{2}	
		
		\begin{itemize}[<default overlay specification>]
			\item<1> Major graphing elements:
				\newline - Lines
				\newline - Shapes
				\newline - Points
			\item<1> Major styling elements:
				\newline - Fill color.
				\newline - Outlines
				\newline - Sizes
			\item<1> Major graphing elements
				\newline - Lines
				\newline - Labels
		\end{itemize}
		
		\begin{figure}
			\centering
			\includegraphics[width=70mm]{img/Graphing2}
		\end{figure}
		
	\end{multicols}
\end{frame}


\begin{frame}{Working with shapes}
	
	\begin{figure}
		\centering
		\includegraphics[width=\linewidth]{img/Shapes}
	\end{figure}
	
\end{frame}


\begin{frame}{Working with colors}
	
	\begin{figure}
		\centering
		\includegraphics[width=\linewidth]{img/Colors}
	\end{figure}
	
\end{frame}


\begin{frame}{Every graph starts from the basics}
	\begin{multicols}{2}	
		
		\leavevmode 	\newline  tw ///
		 	\newline   (qfitci length weight) ///
		 	\newline  (scatter length weight 
		 	\newline  if foreign == 0) ///
		 	\newline  (scatter length weight 
		 	\newline  if foreign == 1) 
		
		\begin{figure}
			\centering
			\includegraphics[width=70mm]{img/Basics}
		\end{figure}
		
	\end{multicols}
\end{frame}


\begin{frame}{With styling, we have a pretty graph}
		
	Options … everywhere
		
		\begin{figure}
			\centering
			\includegraphics[width=135mm]{img/Styling}
		\end{figure}
		
\end{frame}


\begin{frame}{Graphs can be combined and exported}
	\begin{multicols}{2}	
		
		\leavevmode 	\newline  graph export ///

		\leavevmode 	\newline    “filename” /// (.png or .eps)
		\leavevmode		\newline  (, replace
 
		\leavevmode		\newline  With .png, specify “width(1000)” for higher resolution
		\newline  .eps files can scale to any size on most modern software (but hard to preview on older systems)
		
		\begin{figure}
			\centering
			\includegraphics[width=70mm]{img/Styling2}
		\end{figure}
		
	\end{multicols}
\end{frame}


\begin{frame}{DIME Resources (please contribute!)}
	
	\begin{figure}
		\centering
		\includegraphics[width=\linewidth]{img/Resources}
	\end{figure}
	
\end{frame}

%%%%%%%%%%%%%%%% Final thougts section %%%%%%%%%%%%%%%%%%
\begin{frame}{Conclusion}

Thank You!

\vspace{20mm}
For more information or further questions please contact:
\newline Benjamin Daniels (\url{bdaniels@worldbank.org}) 

\end{frame}

%%%%%%%%%%%%%%%%%%%%%%%%%%%%%%%%%%%%%%%%%%% The End
\sectionpic{The End}{img/section_slide}






\end{document} 